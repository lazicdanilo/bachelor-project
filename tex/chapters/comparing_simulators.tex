U nastavku ove sekcije biće opsisana 4 različita softvera za simulaciju
mobilnih robota. Na kraju ove sekcije mozete pronaći finlan odabir softvera
u kome ćemo vršiti simulaciju u nastavku rada.
\subsection{Webots}
Webots je 3D simulaciona platforma razvijena od strane kompanije Cyberbotics.
Jedan je od najkorišćenijih alata u edukacione svrhe, ali i u industriji.
Ovaj alat je podržan na sledećim platformama: Linux, Windows, Mac OS.
Svaki robot se može izmodelirati, pokrenuti i simulirati korišćenjem jednog
(ili kombinacijom više) od sledećih programskih jezika: C, C++, Java, Python, Matlab, ili URBI.\cite{webots-vs-gazebo-2}
Takodje Webots je kompatibilan sa eksternim bibliotekama, kao na primer OpenCV koji je
koriscen u ovom radu u svrhe idnetifikacije objekata i njihovih boja.
Webots odmah nakon instalacije sadrži veliki broj pripremljenih primera koje korisnik
može da modifikuje i da se na njima obučava. Takodje sadrži veliki broj vec modeliranih
robota kao i veliki broj senozora, aktuatora i objekata.
Već postojeći modelirani roboti sadrže prethodno podešene senzore i aktuatore,
neki čak i drajvere robota,tako da korisnik može brzo da ujde u svet mobilne
robotike i da testira svoje ideje, modifikuje postojece drajvere ili doda svoje.
Postojeći senozori variraju od jednostavnijih kao na primer laserski
senzori distance ili LIDAR -a pa do komplikovanijih kao na primer GPS.
Postojeci objekti mogu biti kocke, korisćene za na primer slaganje sa robotskim
rukama, razna vrata, zidovi, stolice i mnogo drugih.
Ovi objekti se mogu iskoristiti za brzo modeliranje sveta u kome će se naš robot
kretati i sa kojim će imati interakciju.
\newline
Svaki od odjekata, senzora, akutuatora, robota ima definisanu masu, faktor trenja sa
različitim podlogama kao i konstante opruge i prigušenja.
Svaki od ovih paramteat se može menjati i podešavati zavsno od potrebe korisnika.
\newline
% Webots nudi mogućnost da se simulacije snime i da se prave slike iz simulacije. Ovo je korisno
% kada je potrebno napraviti...
Webots čuva podatke o svetu koji je korisnik kreirao u .wbt formatu. Fajl kreiran na jednoj
platformi se može koristiti na svim ostalim. Takodje ovaj fajl korisnik može da pročita,
odnosno fajl je textualan tako da ga korisnik moze menjati sa textualnim editorom.
\newline
Webots je besplatan i open source simulator.
%Webots sadži svoj kompajler, možda dodati
%Webots je laksi ya pocetnike cite omparing  Popular  Simulation  Environments  
% in  the  Scope  of  Robotics  andReinforcement Learnin   @article{webots-vs-gazebo-2,

\subsection{Gazebo}
Gazebo nudi moćno, brzo i efikasno okruženje za testiranje autonomnih robota u različitim
uslovima i na različitim terenima. Najviše se koristi za testiranje izbegavanja objekata 
i mašinsku viziju. Takodje je koristan za testiranje algoritama,
samog dizajna robota, obucavanje AI - a itd.
Gazebo koristi više simulatora fizike uključujući ODE (Open Dynamics
Engine), Bullet, Symbody i DART,
tako da renderovanje sveta sa robotom je detaljno i precizno.
Takodje osvetljenje, senke i teksture u svetu koje mogu biti krucijalne u odredjenim primenama
mobilne robotike je ekstremno realistično.
Gazebo takodje nudi veliki broj vec napravljenih i testiranih senozora, kao i
veliki broj već izmodeliranih robota koji su spremni za korišćenje.\cite{gazebo}
\newline
Gazebo je hardverski zahtevnoiji za korišćenje od Webots - a, potrebno mu je više resursa
za stabilan rad\cite{webots-vs-gazebo}\cite{webots-vs-gazebo-2}. Iako u ovom radu neće biti sprovoćene zahtevne simulacije
svakako želmo da prikažemo optimalan način za izvršavanje istih. 
\newline
\newline
Gazebo je besplatan i open source simulator.
\subsection{Unity}
Unity je prvobitno nastao kao platforma za kreiranje video igrica, nakon nekog vremena
poceo je da se korisit u svrhe mobilne robotike. 
\newline
\newline
Unity je besplatan za korišcenje u većini slucajeva, ali nije open source. 


\subsection{NVIDIA Isaac Sim}
NVIDIA Isaac Sim, powered by Omniverse, is a scalable robotics simulation
application and synthetic data generation tool that powers photorealistic,
physically-accurate virtual environments to develop, test,
and manage AI-based robots.
\newline
\newline
NVIDIA Isaac Sim je besplatan i open source simulator.

\subsection{Zakljucak}
Gazebo može biti komplikovan za početnike
Webots je posebno poghodan za korisnike koji po prvi put imaju dodir sa simulacijom 
fizike, ali webots koristi prilagodjenu ODE (Open Dynamics Engine)
verziju i nudi manje podesivih
parametara\cite{webots-vs-gazebo-2}.